\capitulo{CONCLUSÃO E TRABALHOS FUTUROS}\label{sec:conclusao}
\iniciocapitulo

\indent A Educação a Distância vem crescendo com o apoio de ferramentas digitais. Através dos AVA's foi possível aumentar a escala de alcance de usuários, porém acompanhar esses alunos é uma atividade com tendencia a falhas, mas que pode ser auxiliado com o uso de ferramentas computacionais autônomas como Sistemas Multiagentes(SMA's). Através da Mineração de Dados Educacionais, tornou-se possível a descoberta de padrões de comportamento que refletem o desempenho do aluno. Agrupar e interpretar essas informações é de extrema importância para as instituições de ensino.

Neste trabalho foi apresentado o desenvolvimento de um módulo utilizando tecnologia Multiagente e Mineração de Dados, para ser integrado ao SMA desenvolvido pelo grupo GESMA da Universidade Federal do Ceará. Esse módulo tem como finalidade a descoberta de padrões para que seja possível identificar alunos com tendencia a mau desempenho e evasão escolar. Por sua vez, interagindo com os demais agentes do SMA para alterar o cenário dos alunos encontrados com baixo desempenho.

O módulo foi desenvolvido utilizando o \textit{framework JADE} + \textit{JAMDER}, que teve como resultado um agente denominado Agente Controle de Evasão, que possui encapsulado em seus comportamentos processos de descoberta de conhecimento, se beneficiando de aprendizagem supervisionada para a predição dos dados com o algoritmo \textit{Random Forest} e aprendizagem não supervisionada para identificar às classes através de clusterização com o algoritmo \textit{K-Means}, dividindo os alunos em cinco grupos de acordo com seus índices de participação na plataforma \textit{Moodle}. Os cinco grupos citados anteriormente são: alunos com BOA participação, MUITO BOA participação, participação REGULAR, participação RISCO e FORTE RISCO. Identificando os alunos correspondentes a esses grupos, foi possível realizar a elaboração de um modelo de dados para predição, o qual foi utilizado para acompanhar o desempenho dos alunos ao decorrer do semestre letivo, tendo como finalidade remediar os possíveis alunos com tendencia a evasão, a fim de os ajudarem a melhorar o desempenho, através dos comportamentos do Agente Companheiro de Aprendizagem implementado no SMA.

Através da clusterização, foi possível observar a imensa quantidade de alunos que estão com desempenho ruim ou muito ruim nos cursos. Os dados desses alunos totalizam 51\% de todo o \textit{dataset}, o que é algo preocupante, sendo possível concluir que é necessário melhorar o acompanhamento dos alunos, dessa forma utilizando sistemas computacionais autônomos como Sistemas Multiagentes e Mineração de Dados Educacionais.

O presente trabalho resultou em um módulo para predição de alunos com tendência a evasão na plataforma \textit{Moodle}, além dele ser utilizado como uma parte integrante do SMA desenvolvido pelo grupo GESMA, ele também pode ser utilizado individualmente, sendo necessário o desenvolvimento de uma interface gráfica para a interpretação visual dos dados. Porém, o \textit{Data Mart} está organizado de uma forma que está pronta para fornecer diversos relatórios.

Além da definição e desenvolvimento do processo de KDD para o presente contexto, o trabalho contribuiu com uma analise de algoritmos de aprendizagem supervisionada, comparando-os em relação a acurácia, o resultado foi obtido através da técnica \textit{cross-validation} no modelo de treinamento gerado após o processo de clusterização para a descoberta das classes, dos quais foi possível observar dentre os selecionados, o mais eficiente foi o \textit{Random Forest}.

O SMA não foi testado em ambiente real com o módulo de Controle de Evasão, assim não sendo possível observar mudanças reais em relação ao desempenho dos alunos, porém com o processo de KDD validado, tendo uma acurácia significativa de 98,27\% de acerto, foi possível identificar precocemente alunos que tendem a evasão, dessa forma acionando os Agentes necessários para mudar esse cenário. 

Como trabalhos futuros podemos citar uma analise para fragmentar em mais Agentes as funcionalidades contidas nos comportamentos do Agente Controle de Evasão, o desenvolvimento de uma interface gráfica para configuração dos parâmetros necessários para o funcionamento do módulo, como também a visualização mais detalhada das informações que podem ser obtidas através do \textit{Data Mart} que foi definido no presente trabalho. E por fim, o estudo da viabilidade da migração do \textit{Data Mart} do banco de dados relacional para um banco de dados não relacional, visando o ganho de desempenho para a manipulação de imensas cargas de dados.
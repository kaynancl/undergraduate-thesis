\capitulo{INTRODUÇÃO}
\iniciocapitulo

A Educação a Distância (EAD) é uma modalidade de ensino que tem como principal ferramenta o Ambiente Virtual de Aprendizagem (AVA), onde professores, alunos e tutores podem interagir de forma direta ou indireta. Com grande flexibilidade de tempo e compromisso com essa modalidade de ensino, o discente necessita manter o foco, contato com os tutores e organização em seus horários para estudo, porém, não é possível desconsiderar a vida pessoal do indivíduo e é devido a essa flexibilidade que a EAD permite que apareçam alguns problemas em relação ao desempenho do aluno no decorrer do curso. Muitos alunos chegam a desistir/evadir por causa de problemas financeiros, falta de tempo para o comprometimento com os estudos, falta de material didático auxiliar disponível no AVA, falta de profissionalismo dos tutores, e entre outros fatores \cite{cavalcantimineraccao}.

Devido ao aumento de alunos na EAD, gerenciar seus processos de aprendizagem com qualidade de interação e de acompanhamento dentro de um AVA, visando o êxito e a permanência dos alunos nos seus respectivos cursos, é uma tarefa que exige cada vez mais dos professores. Os dados gerados nas interações entre professores e alunos, dos alunos entre si e deles com os recursos disponibilizados no AVA, são volumosos e pouco explorados, podendo conter informações uteis para a instituição, porém reuni-los e interpreta-los é uma atividade complexa e exaustiva \cite{kampff2009mineraccao}.

Acompanhar o aluno no decorrer do curso é fundamental para o êxito no curso. Com um bom sistema de acompanhamento e avaliação, é possível observar características que representam suas dificuldades e assim poderia ser oferecido o tipo adequado de ajuda. Existem algumas soluções desenvolvidas para auxiliar o acompanhamento do desempenho de alunos em um AVA, como por exemplo o Sistema Multiagente desenvolvido pelo Grupo de Estudo de Engenharia de Software em Sistemas Multiagente (GESMA) da Universidade Federal do Ceará (UFC) Campus Quixadá que integra a Universidade Aberta do Brasil (UAB) da Universidade Estadual do Ceará (UECE), denominado \textit{SMA Moodle} \cite{gonccalvesabordagem}. Este sistema tem como principal objetivo auxiliar o acompanhamento de alunos no AVA \textit{Moodle}\footnote{Disponível em: https://moodle.org}, plataforma de Educação a Distância utilizada mundialmente e pela Universidade Estadual do Ceará (UECE), que utiliza a modalidade semi presencial, cujo os alunos cujo em alguns casos os alunos realizam atividades presencias. O sistema é composto por um conjunto de agentes, onde cada agente através de seus compromissos, se responsabiliza por uma parte do AVA. Algumas das funcionalidades desse SMA são: acompanhar o desempenho do aluno durante os cursos matriculados, acompanhar as atividades dos tutores dos respectivos cursos, criar grupos de alunos de acordo com o perfil e temas de interesse e enviar materiais de apoio aos alunos e tutores.

Este trabalho teve como principal objetivo o desenvolvimento de um módulo que foi integrado ao SMA. Este modulo será responsável pela identificação prévia de características que representam comportamentos que podem levar o aluno a evadir ou a ter mau desempenho no curso. Para que isso fosse possível, foram analisados dados históricos dos alunos da UECE. Esses dados sofreram um processo de clusterização para dividir em grupos os perfis dos alunos, e através de classificação foi possível prever o desempenho dos alunos para que fossem ajudados pelo SMA. Com esses valores de desempenho identificados, informações são repassadas aos demais agentes do sistema para que decisões sejam tomadas em conjunto por eles. Antecipar a identificação desses perfis é de grande utilidade e interesse das instituições de ensino que têm como método de ensino a EAD, pois tanto os docentes poderão remediar da melhor forma a situação, como os discentes terão um acompanhamento mais adequado no decorrer do curso, e, por sua vez, diminuindo a quantidade de alunos que podem evadir, resultando no aumento de concludentes dos cursos.

O intuito desse projeto é propor uma abordagem que irá somar com o SMA desenvolvido pelo grupo GESMA, aumentando sua utilidade e funcionalidades para o corpo docente e discente que venham a utilizar o AVA \textit{Moodle}.

Esse trabalho está divido nas seguintes seções: a Seção \ref{sec:fundamentacao} descreve os principais conceitos usados no trabalho; a Seção \ref{sec:trabalhosrel} apresenta alguns trabalhos relacionados; a Seção \ref{sec:metodologia} descreve o experimento e os resultados, bem como a análise dos mesmos; e a Seção \ref{sec:conclusao} conclui e pincela sobre os trabalhos futuros para tornar o modulo mais robusto e preciso para o acompanhamento dos alunos.